\documentclass[]{raa} 

\usepackage{aas_macros}
\usepackage{graphicx,times}
\usepackage{natbib}
\usepackage{amssymb,amsmath}
\bibpunct{(}{)}{;}{a}{}{,}

\usepackage{hyperref}
\hypersetup{colorlinks = true, linkcolor = green, anchorcolor = red, citecolor = blue, filecolor = red, pagecolor = red, urlcolor = red}

\newcommand{\diff}{{\rm d}}
\newcommand{\cpm}{{\small CUBEP$^3$M}}
\newcommand{\smjustify}{\hspace{-0.1cm}}
\def\smwidth{0.48\textwidth}
\def\smhwidth{0.48\textwidth}

\begin{document}

\title{Searching 21-cm Absorption Systems in Chinese Radio Telescopes}

\author{
Hao-Ran Yu\inst{1,2},
Ue-Li Pen\inst{2,3,4,5},
Tong-Jie Zhang\inst{6},
Di Li\inst{7,8} and
Xuelei Chen\inst{9}
}

\institute{
Kavli Institute for Astronomy \& Astrophysics, Peking University, Beijing 100871, China;\\ \and
Canadian Institute for Theoretical Astrophysics, University of Toronto, Toronto, M5S 3H8, ON, Canada;\\ \and
Dunlap Institute for Astronomy and Astrophysics, University of Toronto, Toronto, M5S 3H4, ON, Canada;\\ \and
Canadian Institute for Advanced Research, Program in Cosmology and Gravitation;\\ \and
Perimeter Institute for Theoretical Physics, Waterloo, ON, N2L 2Y5, Canada;\\ \and
Department of Astronomy, Beijing Normal University, Beijing, 100875, China;\\ \and
National Astronomical Observatories, Chinese Academy of Sciences, Beijing, China;\\ \and
Key Laboratory of Radio Astronomy, Chinese Academy of Sciences, Beijing, China;\\ \and
Key Laboratory for Computational Astrophysics, National Astronomical Observatories, Chinese Academy of Sciences, Beijing, 100012, China.\\
}

\abstract{
Neutral hydrogen clouds are known to exist in the Universe, however their spatial
distributions and physical properties are poorly understood. Such missing information
can be studied by the Chinese new generation radio telescopes by a blind searching
of 21-cm absorption systems. We forecast the abilities of surveys of 21-cm absorption
systems by two representative radio telescopes --
Five-hundred-meter Aperture Spherical radio Telescope (FAST) and
Tianlai. The result shows that, in a few of years term,
these telescopes with either high sensitivity (FAST)
or wide field of view (Tianlai) can discover orders of magnitudes
more 21-cm absorption systems, than the cumulative discoveries
in the past 50 years. 
\keywords{}
}

\authorrunning{Yu et al.}
\titlerunning{Searching 21-cm absorption systems}

\maketitle

\section{Introduction}
21-cm absorption systems can be used to directly measure the cosmic acceleration
via Sandage-Loeb (SL) effect \citep{1962ApJ...136..319S,1998ApJ...499L.111L}.
The current best constraint of this acceleration if given by 
\cite{2012ApJ...761L..26D}. A proposal
on CHIME-like telescope is given by \cite{2014PhRvL.113d1303Y},
in which the minor redshift drift could be measured over decades.
The uncertainties are from the poorly understood neutral hydrogen
clouds. Path finder surveys, taking less than a year, are needed
to improve our understandings of spatial distribution and physical
properties of neutral hydrogen clouds.
This can be done by the Chinese new generation radio telescopes --
Five-hundred-meter Aperture Spherical radio Telescope (FAST)
\citep{2016RaSc...51.1060L} and
Tianlai \citep{2012IJMPS..12..256C}.


\section{Sensitivity}

NRAO VLA Sky Survey (NVSS)\footnote{http://www.cv.nrao.edu/nvss/}.

Recent studies show that the number density of absorbers to be $\diff N/\diff z\sim 0.45$
\citep{2005ARA&A..43..861W,2007ASSP....3..501Z}.

Recent discoveries of 21-cm absorption systems have about $20\%$ fractional depth
\citep{2015MNRAS.453.1249A,2015MNRAS.453.1268Z}.

\section{FAST estimation}

Parameters:

Result: 90 per month.

\section{Tianlai estimation}

Parameters:

Result: 80 per year.

\section{Conclusion}

\begin{acknowledgements}
HRY acknowledges General Financial Grant No.2015M570884 and
Special Financial Grant No. 2016T90009 from the China
Postdoctoral Science Foundation.
HRY and ULP acknowledge the support of the
National Science and Engineering Research Council of Canada.
\end{acknowledgements}

\bibliographystyle{raa}
\bibliography{haoran_ref}

\end{document}
